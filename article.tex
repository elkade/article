\documentclass[twoside,twocolumn]{article}

\usepackage{blindtext} % Package to generate dummy text throughout this template 

\usepackage[sc]{mathpazo} % Use the Palatino font
\usepackage[T1]{fontenc} % Use 8-bit encoding that has 256 glyphs
\linespread{1.05} % Line spacing - Palatino needs more space between lines
\usepackage{microtype} % Slightly tweak font spacing for aesthetics
\usepackage[utf8]{inputenc}
\usepackage[english]{babel} % Language hyphenation and typographical rules

\usepackage[hmarginratio=1:1,top=32mm,columnsep=20pt]{geometry} % Document margins
\usepackage[hang, small,labelfont=bf,up,textfont=it,up]{caption} % Custom captions under/above floats in tables or figures
\usepackage{booktabs} % Horizontal rules in tables

\usepackage{lettrine} % The lettrine is the first enlarged letter at the beginning of the text

\usepackage{enumitem} % Customized lists
\setlist[itemize]{noitemsep} % Make itemize lists more compact

\usepackage{abstract} % Allows abstract customization
\renewcommand{\abstractnamefont}{\normalfont\bfseries} % Set the "Abstract" text to bold
\renewcommand{\abstracttextfont}{\normalfont\small\itshape} % Set the abstract itself to small italic text

\usepackage{titlesec} % Allows customization of titles
\renewcommand\thesection{\Roman{section}} % Roman numerals for the sections
\renewcommand\thesubsection{\roman{subsection}} % roman numerals for subsections
\titleformat{\section}[block]{\large\scshape\centering}{\thesection.}{1em}{} % Change the look of the section titles
\titleformat{\subsection}[block]{\large}{\thesubsection.}{1em}{} % Change the look of the section titles

\usepackage{fancyhdr} % Headers and footers
\pagestyle{fancy} % All pages have headers and footers
\fancyhead{} % Blank out the default header
\fancyfoot{} % Blank out the default footer
\fancyhead[C]{Running title $\bullet$ May 2016 $\bullet$ Vol. XXI, No. 1} % Custom header text
\fancyfoot[RO,LE]{\thepage} % Custom footer text

\usepackage{titling} % Customizing the title section

\usepackage{hyperref} % For hyperlinks in the PDF

%----------------------------------------------------------------------------------------
%	TITLE SECTION
%----------------------------------------------------------------------------------------

\setlength{\droptitle}{-4\baselineskip} % Move the title up

\pretitle{\begin{center}\Huge\bfseries} % Article title formatting
	\posttitle{\end{center}} % Article title closing formatting
\title{Content-based recommendations in~e-commerce services} % Article title
\author{%
	\textsc{John Smith}\thanks{A thank you or further information} \\[1ex] % Your name
	\normalsize Warsaw University of Technology \\ % Your institution
	\normalsize Faculty of Mathematics and Information Science \\
	\normalsize ul. Koszykowa 75 \\
	\normalsize 00-662 Warsaw, Poland \\
	\normalsize \href{mailto:john@smith.com}{john@smith.com} % Your email address
	%\and % Uncomment if 2 authors are required, duplicate these 4 lines if more
	%\textsc{Jane Smith}\thanks{Corresponding author} \\[1ex] % Second author's name
	%\normalsize University of Utah \\ % Second author's institution
	%\normalsize \href{mailto:jane@smith.com}{jane@smith.com} % Second author's email address
}
\date{\today} % Leave empty to omit a date
\renewcommand{\maketitlehookd}{%
	\begin{abstract}
		The subject of this paper focuses on issues of determining a semantic similarity between text
		documents and recommendations of similar documents. A detailed problem comes from the online
		auction site Allegro, which has a section of articles describing products available on the platform.
		This section offers a recommendation system for similar textual articles based on their
		content. The aim of this paper is to investigate a possibility of improving the existing recommendation
		system using semantic text analysis methods.
		In this paper, I adapt some state-of-the-art methods for determining a similarity between
		text documents to the above problem, I introduce measures to evaluate a performance of these
		methods and analyze possibilities of using them in the real system.
	\end{abstract}
	{\bf Keywords:} recommendations, natural language processing, word embedding, semantics, allegro
	
}

%----------------------------------------------------------------------------------------

\begin{document}
	
	% Print the title
	\maketitle

	\section{Introduction}
	
	Recommendation systems are often part of web services. 
	
	The key issue of recommendations generation is how suggested items are relevant to these which the user is interested in. 
	
	We can divide recommendation systems into two groups: collaborative and  content-based filtering. The first one assumes that user is likely to be interested in items which also users similar to s/he were interested in. In this paper we are focusing on the second group in which recommended items are similar to these that the user liked so far.
	
	In this paper we strive to check if newly proposed word embeddings methods are able to replace recently used method based on Elasticsearch query.
	
	Allegro - the biggest marketplace platform in Eastern Europe contains a section presenting text articles concerning products available vie the platform. Currently there is a list of links to articles similar to given one. 
	% cel badania - sprawdzić czy można zastąpić używaną do tej pory metodę opartą o zapytanie elasticsearchowe
	
	\lettrine[nindent=0em,lines=3]{L} orem ipsum dolor sit amet, consectetur adipiscing elit.
	\blindtext % Dummy text
	
	\blindtext % Dummy text
	
	%------------------------------------------------
	
	\section{Methods}
	
	Maecenas sed ultricies felis. Sed imperdiet dictum arcu a egestas. 
	\begin{itemize}
		\item Donec dolor arcu, rutrum id molestie in, viverra sed diam
		\item Curabitur feugiat
		\item turpis sed auctor facilisis
		\item arcu eros accumsan lorem, at posuere mi diam sit amet tortor
		\item Fusce fermentum, mi sit amet euismod rutrum
		\item sem lorem molestie diam, iaculis aliquet sapien tortor non nisi
		\item Pellentesque bibendum pretium aliquet
	\end{itemize}
	\blindtext % Dummy text
	
	Text requiring further explanation\footnote{Example footnote}.
	
	%------------------------------------------------
	
	\section{Dataset}
	
	Given dataset consists of 20000 textual articles concerning different products available via Allegro platform.
	
	\section{Results and discussion}
	
	% metody ewaluacji i wyniki
	
	\begin{table}
		\caption{Example table}
		\centering
		\begin{tabular}{llr}
			\toprule
			\multicolumn{2}{c}{Name} \\
			\cmidrule(r){1-2}
			First name & Last Name & Grade \\
			\midrule
			John & Doe & $7.5$ \\
			Richard & Miles & $2$ \\
			\bottomrule
		\end{tabular}
	\end{table}
	
	\blindtext % Dummy text
	
	\begin{equation}
	\label{eq:emc}
	e = mc^2
	\end{equation}
	
	\blindtext % Dummy text
	
	%------------------------------------------------
	
	\section{Conclusion}
	
	\subsection{Subsection One}
	
	A statement requiring citation \cite{Figueredo:2009dg}.
	\blindtext % Dummy text
	
	\subsection{Subsection Two}
	
	\blindtext % Dummy text
	
	\begin{thebibliography}{99} % Bibliography - this is intentionally simple in this template
		
		\bibitem[Figueredo and Wolf, 2009]{Figueredo:2009dg}
		Figueredo, A.~J. and Wolf, P. S.~A. (2009).
		\newblock Assortative pairing and life history strategy - a cross-cultural
		study.
		\newblock {\em Human Nature}, 20:317--330.
		
	\end{thebibliography}
	
	%----------------------------------------------------------------------------------------
	
\end{document}
.